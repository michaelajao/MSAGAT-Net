\section{Visualization and Figure Analysis}
\label{sec:figure_analysis}

This section provides an in-depth analysis of the visualizations and figures that illustrate key aspects of MSTGAT-Net's performance and behavior. These visualizations help elucidate the model's learning process, the patterns it captures, and its performance characteristics across different contexts.

\subsection{Performance Comparison Visualizations}

\subsubsection{RMSE vs. Forecast Horizon}

Figure~\ref{fig:rmse_vs_horizon} illustrates how forecast error (RMSE) increases with the forecast horizon for MSTGAT-Net and baseline models. The key observations from this figure include:

\begin{itemize}
    \item All models show increasing error with longer forecast horizons, which is expected in time series forecasting due to the inherent uncertainty of more distant predictions.
    
    \item The error growth rate is notably lower for MSTGAT-Net compared to baseline models, demonstrating its superior capacity to maintain accuracy for longer-term forecasts.
    
    \item The gap between MSTGAT-Net and the next-best model (Graph WaveNet) widens as the forecast horizon increases, highlighting MSTGAT-Net's particular strength in medium to long-term forecasting.
    
    \item Traditional time series models (HA, VAR) show the steepest error increase with horizon, underscoring the importance of spatial dependency modeling for epidemic forecasting.
\end{itemize}

\subsubsection{PCC vs. Forecast Horizon}

Figure~\ref{fig:pcc_vs_horizon} shows how the Pearson Correlation Coefficient (PCC) changes with the forecast horizon. This metric focuses on pattern matching rather than absolute error values, providing complementary insights:

\begin{itemize}
    \item MSTGAT-Net maintains higher correlation with ground truth even at longer horizons, demonstrating its ability to capture future temporal patterns.
    
    \item The PCC advantage of MSTGAT-Net over baselines grows at longer horizons, reinforcing its particular strength in forecasting distant time steps.
    
    \item Deep learning models maintain substantially higher PCCs than statistical models across all horizons, highlighting the value of their representational capacity.
\end{itemize}

\subsubsection{Performance Heatmaps}

Figure~\ref{fig:performance_heatmaps} visualizes performance metrics (RMSE, MAE, PCC, R²) across datasets and forecast horizons as color-coded heatmaps. This visualization reveals:

\begin{itemize}
    \item Clear patterns of performance variation across datasets, with the NHS dataset showing the best overall performance and the Regional dataset presenting the greatest challenge.
    
    \item The expected degradation of all metrics with increasing forecast horizon, but with MSTGAT-Net maintaining reasonable performance even at the longest horizons tested.
    
    \item Interesting dataset-specific patterns, such as the UK LTLA dataset showing steeper performance degradation with horizon length, likely due to its fine-grained spatial resolution capturing more volatile local patterns.
\end{itemize}

\subsubsection{Performance Radar Chart}

Figure~\ref{fig:performance_radar} presents a radar chart comparing MSTGAT-Net against baseline models across multiple metrics. This multi-dimensional visualization shows that:

\begin{itemize}
    \item MSTGAT-Net consistently outperforms all baselines across all metrics, with no single baseline model proving superior in any dimension.
    
    \item The performance advantage is most pronounced for PCC and R², suggesting that MSTGAT-Net particularly excels at capturing the underlying patterns and trends in the data.
    
    \item Graph WaveNet consistently emerges as the second-best model, while simpler models like HA and VAR show substantially worse performance across all metrics.
\end{itemize}

\subsection{Component Analysis Visualizations}

\subsubsection{Component Importance Heatmap}

Figure~\ref{fig:component_importance_heatmap} visualizes the relative importance of each component through a heatmap showing performance degradation when the respective component is removed. This visualization reveals:

\begin{itemize}
    \item The Adaptive Graph Attention Module (AGAM) consistently shows the highest importance across most datasets and horizons.
    
    \item The Dilated Multi-Scale Temporal Module (DMTM) shows moderate but consistent importance across all configurations.
    
    \item The Progressive Prediction Module (PPM) shows increasing importance with horizon length, consistent with its role in mitigating error accumulation.
    
    \item Dataset-specific patterns in component importance, such as AGAM being particularly crucial for the Regional dataset and DMTM showing higher relative importance for the UK LTLA dataset.
\end{itemize}

\subsubsection{Component Contribution Bar Charts}

Figure~\ref{fig:component_contribution_rmse} and Figure~\ref{fig:component_contribution_pcc} present stacked bar charts showing the contribution of each component to RMSE reduction and PCC improvement, respectively. These visualizations highlight:

\begin{itemize}
    \item The cumulative nature of performance improvement, with each component adding meaningful value.
    
    \item The relatively consistent ranking of component importance: AGAM > DMTM > PPM for most configurations.
    
    \item Horizon-dependent contribution patterns, particularly for PPM, which contributes proportionally more at longer horizons.
\end{itemize}

\subsubsection{Component Impact by Dataset}

Figures~\ref{fig:component_impact_japan_h5}, \ref{fig:component_impact_region785_h5}, and others show the performance impact of removing each component for specific dataset-horizon combinations. These detailed views reveal:

\begin{itemize}
    \item Subtle variations in component importance across different geographical contexts.
    
    \item The consistency of AGAM's importance across diverse spatial scales, from the 47 prefectures of Japan to the 785 regions of the Regional dataset.
    
    \item The complementary nature of the components, with their combined removal leading to substantially worse performance than any single component removal.
\end{itemize}

\subsection{Learned Pattern Visualizations}

\subsubsection{Attention Matrix Visualization}

Figure~\ref{fig:attention_patterns_japan} visualizes the learned attention weights for the Japan dataset, representing the spatial dependencies discovered by the model. Key observations include:

\begin{itemize}
    \item Strong attention weights between geographically adjacent prefectures, despite the model not being explicitly provided with geographical information.
    
    \item Hub-like patterns centered on major urban areas like Tokyo, Osaka, and Nagoya, reflecting their roles as transportation and population centers.
    
    \item Non-trivial long-distance connections that align with major transportation routes or regions with similar epidemic patterns.
    
    \item Temporal evolution of attention patterns, with denser connectivity during spreading phases and more isolated patterns during containment periods.
\end{itemize}

These visualizations demonstrate MSTGAT-Net's ability to learn meaningful spatial relationships from data, rather than relying on pre-defined structures that may not capture the full complexity of epidemic transmission patterns.

\subsubsection{Scale Weight Visualization}

Figure~\ref{fig:scale_weights} illustrates the learned weights assigned to different temporal scales across regions. This visualization reveals:

\begin{itemize}
    \item Region-specific preferences for temporal scales, with urban regions favoring shorter scales and rural regions emphasizing longer scales.
    
    \item Temporal adaptation of scale weights as the epidemic progresses through different phases.
    
    \item Correlation between scale preferences and local epidemic characteristics, suggesting that the model adapts its temporal processing to the specific dynamics of each region.
\end{itemize}

\subsubsection{Gate Value Visualization}

Figure~\ref{fig:gate_values_horizon} shows the values of the gates in the Progressive Prediction Module across the forecast horizon. This visualization illustrates:

\begin{itemize}
    \item A general decreasing trend in gate values with time step, indicating greater reliance on recent observations for near-term predictions and more on model predictions for longer-term forecasts.
    
    \item Regional variation in gating strategies, with some regions maintaining higher gate values (more observation influence) throughout the horizon.
    
    \item Temporal patterns in gate values during different epidemic phases, with higher gate values during volatile periods, suggesting adaptive adjustment of the prediction strategy based on current dynamics.
\end{itemize}

\subsection{Case Study Visualizations}

\subsubsection{Regime Change Response}

Figure~\ref{fig:regime_change} compares MSTGAT-Net against baseline models during a sudden case surge in Japan (August 2021). This visualization highlights:

\begin{itemize}
    \item MSTGAT-Net's superior ability to quickly adapt to the changing trend, capturing the onset of the surge earlier than baseline models.
    
    \item The growing advantage of MSTGAT-Net as the surge progresses, demonstrating its ability to maintain accuracy during rapidly changing conditions.
    
    \item Qualitative differences in prediction patterns, with MSTGAT-Net producing more realistic predictions that capture both the trend and short-term fluctuations.
\end{itemize}

\subsubsection{Regional Type Comparison}

Figure~\ref{fig:region_type_comparison} compares model performance across high, medium, and low population density regions. This visualization reveals:

\begin{itemize}
    \item MSTGAT-Net's consistent advantage across all region types, but with varying margins of improvement.
    
    \item Larger performance gains in high-density regions, highlighting the particular value of adaptive graph attention and multi-scale temporal modeling in complex urban environments.
    
    \item Different error patterns across region types, with low-density regions showing more consistent but generally lower case counts and high-density regions exhibiting more volatile patterns.
\end{itemize}

\subsection{Loss Curves and Training Dynamics}

The loss curves in the `figures` directory document the training dynamics of different model variants across datasets and forecast horizons. Analysis of these curves reveals:

\begin{itemize}
    \item The full MSTGAT-Net consistently achieves lower validation loss than ablated variants, confirming the value of each component.
    
    \item Different convergence patterns across datasets, with more complex datasets (Regional, UK LTLA) requiring more epochs to converge.
    
    \item Ablation-specific patterns, with AGAM removal often leading to higher initial loss and slower convergence, highlighting its importance for efficient learning.
    
    \item Horizon-dependent training dynamics, with longer horizons generally showing higher absolute loss values but similar relative patterns between model variants.
\end{itemize}

\subsection{Summary of Visual Analysis}

The comprehensive set of visualizations provides compelling evidence for MSTGAT-Net's effectiveness across diverse forecasting contexts. The visualizations highlight not only the quantitative performance advantages but also qualitative insights into how the model learns and adapts to complex spatiotemporal patterns.

Key insights from the visual analysis include:

\begin{itemize}
    \item MSTGAT-Net consistently outperforms baselines across all metrics, datasets, and forecast horizons.
    
    \item Each component makes meaningful and complementary contributions to overall performance.
    
    \item The model learns interpretable spatial attention patterns and adaptive temporal scale preferences without explicit guidance.
    
    \item The performance advantage is particularly pronounced during regime changes and for longer forecast horizons.
    
    \item Different geographical contexts benefit from the model's adaptivity in different ways, with complex urban environments showing the largest improvements.
\end{itemize}

These visualizations not only validate the architectural choices in MSTGAT-Net but also provide insights into how the model works, enhancing both its scientific contribution and its potential utility for practical epidemic forecasting applications.
