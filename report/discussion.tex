\section{Discussion and Analysis}
\label{sec:discussion}

In this section, we provide a comprehensive analysis of the results presented earlier, exploring the implications of our experiments, the relative contributions of the model's components, and the observed performance patterns across different datasets and forecast horizons.

\subsection{Comparative Performance Analysis}

MSTGAT-Net consistently outperformed all baseline models across all four datasets (Japan, Regional, UK LTLA, and Spain) and across all forecast horizons tested. The performance advantage was particularly pronounced for medium and long-term predictions (5-15 days ahead), where traditional models struggled with compounding errors. 

\subsubsection{Horizon-Specific Performance}

As shown in Figure~\ref{fig:rmse_vs_horizon}, the performance advantage of MSTGAT-Net over baseline models increases with the forecast horizon. While Graph WaveNet and ASTGCN exhibited competitive performance for short-term forecasting (3-day horizon), their accuracy deteriorated more rapidly than MSTGAT-Net as the forecast horizon increased. For instance, on the Japan dataset, MSTGAT-Net maintained a Pearson Correlation Coefficient (PCC) above 0.80 even for a 10-day horizon, while Graph WaveNet's PCC dropped below 0.75 at the same horizon.

This observed pattern can be attributed to three key design elements of MSTGAT-Net:

\begin{enumerate}
    \item \textbf{Adaptive Graph Attention}: By dynamically adjusting spatial relationships based on evolving epidemic patterns, MSTGAT-Net captures shifting patterns of spatial dependency that static graph-based models cannot.
    
    \item \textbf{Multi-Scale Temporal Processing}: The dilated convolutions at multiple scales allow MSTGAT-Net to simultaneously model both immediate temporal dependencies and longer-term patterns, essential for maintaining accuracy over extended horizons.
    
    \item \textbf{Progressive Prediction Strategy}: The adaptive blending mechanism mitigates error accumulation by intelligently combining model predictions with recent observations, particularly valuable for longer forecast horizons.
\end{enumerate}

\subsubsection{Dataset-Specific Performance}

Performance varied across datasets in ways that align with their inherent characteristics. The model achieved the highest accuracy (in terms of both RMSE and PCC) on the NHS dataset, which contains more regular and predictable patterns. The Japan dataset, with its 47 prefectures exhibiting strong interconnections, saw the second-best performance. The more challenging Regional dataset with 785 regions and the UK LTLA dataset with its fine-grained spatial resolution exhibited lower absolute performance but still substantially outperformed baseline models.

Figure~\ref{fig:performance_heatmaps} illustrates these dataset-specific patterns across all metrics (RMSE, MAE, PCC, and R²). The model's adaptability to different spatial scales and data characteristics supports its applicability across diverse epidemic monitoring contexts.

\subsection{Ablation Analysis and Component Contributions}

The ablation studies provide quantitative evidence regarding the contribution of each core component of MSTGAT-Net. As illustrated in Figure~\ref{fig:component_importance_heatmap}, the relative importance of each component varies somewhat by dataset and forecast horizon, but several consistent patterns emerge.

\subsubsection{Adaptive Graph Attention Module (AGAM)}

Removing the Adaptive Graph Attention Module consistently resulted in the largest performance degradation across all datasets and horizons. This decline was most pronounced for datasets with complex spatial relationships (e.g., the Regional dataset) and for longer forecast horizons, where accurately modeling evolving spatial dependencies becomes increasingly important.

For the Japan dataset with a 5-day horizon, removing AGAM resulted in a 12.3\% increase in MAE and a 3.99\% increase in RMSE, highlighting its critical role in capturing meaningful spatial relationships. The visualization of learned attention matrices (Figure~\ref{fig:attention_patterns_japan}) shows how AGAM captures non-trivial connections between regions that align with transportation networks and population movement patterns, despite not being explicitly provided with this information.

\subsubsection{Dilated Multi-Scale Temporal Module (DMTM)}

The Dilated Multi-Scale Temporal Module provided the second largest contribution to model performance on average. Its importance was particularly evident in datasets with complex temporal dynamics and for longer forecast horizons. For instance, in the Japan dataset with a 10-day horizon, removing DMTM caused a 7.2\% increase in MAE and a significant reduction in PCC from 0.81 to 0.76.

Analysis of the scale weights learned by this module revealed that different regions prioritize different temporal scales depending on their local epidemic characteristics. Urban centers tend to assign higher weights to shorter temporal scales, reflecting their more volatile patterns, while rural regions often emphasize longer scales, consistent with their more gradual progression.

\subsubsection{Progressive Prediction Module (PPM)}

While the Progressive Prediction Module generally shows a smaller overall contribution than the other two components, its importance increases significantly with the forecast horizon. For a 3-day horizon on the Japan dataset, removing PPM resulted in only a 2.1\% increase in MAE, but for a 10-day horizon, this impact grew to 8.4\%.

This pattern aligns with the module's design purpose: mitigating error accumulation in longer-term forecasts. The visualization of gate values across the forecast horizon (Figure~\ref{fig:gate_values_horizon}) confirms that the model learns to rely more heavily on recent observations for the immediate future while gradually transitioning to model-based predictions for later time steps.

\subsubsection{Component Interactions}

An interesting finding from the ablation studies is that the components exhibit synergistic interactions. The performance degradation from removing all three components is greater than the sum of individual component removals, suggesting that these modules complement each other. This complementarity is particularly evident between the AGAM and DMTM modules, where the combined removal resulted in substantially worse performance than expected from their individual contributions.

\subsection{Spatial Attention Analysis}

The visualization of learned attention patterns reveals several noteworthy characteristics of MSTGAT-Net's spatial modeling capabilities.

\subsubsection{Emergent Geographical Awareness}

Despite not being explicitly provided with geographical information, MSTGAT-Net learned attention patterns that closely align with actual geographical proximity and transportation connectivity. In the Japan dataset, strong connections emerged between geographically adjacent prefectures, and major urban centers (Tokyo, Osaka, Nagoya) naturally developed as influential nodes with high connectivity to other regions.

\subsubsection{Temporal Evolution of Spatial Dependencies}

The attention patterns were not static but evolved during different phases of the epidemic. During the early spreading phase of a new wave, the model generated denser connectivity patterns, capturing the widespread transmission. In contrast, during containment or decline phases, more isolated patterns emerged, reflecting the localized nature of remaining cases.

\subsubsection{Discovery of Non-Trivial Connections}

Beyond obvious geographical relationships, MSTGAT-Net identified non-trivial connections between distant regions that share similar temporal patterns or are connected by major transportation routes. For instance, in the Japan dataset, the model learned strong connections between Tokyo and tourist destinations or business centers, even when geographically distant.

\subsection{Multi-Scale Temporal Feature Analysis}

The analysis of the multi-scale temporal features revealed important insights about how different temporal patterns contribute to the forecasting process.

\subsubsection{Region-Specific Scale Preferences}

Different regions exhibited distinct preferences for temporal scales, with these preferences aligning with local epidemic characteristics. Urban regions, with their more volatile patterns driven by higher population density and mobility, tended to prioritize shorter temporal scales. In contrast, rural regions with more gradual epidemic progression assigned higher weights to longer temporal scales, capturing slower-changing patterns.

\subsubsection{Adaptive Scale Transitions}

The scale weights exhibited temporal evolution as the epidemic progressed through different phases. During the onset of a new wave, shorter temporal scales received higher weights, allowing the model to quickly adapt to changing dynamics. During stable or decline phases, longer scales became more important, capturing the more gradual changes characteristic of these periods.

\subsubsection{Scale Importance Across Horizons}

The relative importance of different temporal scales varied with the forecast horizon. For short-term predictions (3-day), the finest temporal scale dominated, while for longer horizons (10-day, 15-day), coarser scales received higher weights. This pattern reflects the increasing importance of capturing longer-term trends as the forecast horizon extends.

\subsection{Performance During Regime Changes}

A particularly notable strength of MSTGAT-Net is its performance during significant regime changes, such as the onset of new epidemic waves or the implementation of intervention measures.

\subsubsection{Case Study: Wave Onset}

During the sudden case surge in August 2021 in Japan, MSTGAT-Net adapted more quickly to the changing dynamics, with a 23.5\% lower MAE during the first week of the surge compared to Graph WaveNet. This adaptability stems from multiple architectural elements:

\begin{itemize}
    \item The PPM's adaptive blending of model predictions with recent observations allowed for rapid adjustment to the new trend.
    \item The DMTM's multi-scale processing captured emerging patterns at different temporal granularities simultaneously.
    \item The AGAM quickly adapted to changes in inter-regional transmission patterns as the wave spread.
\end{itemize}

\subsubsection{Intervention Response}

Similarly, when examining periods following major intervention implementations (e.g., lockdowns or mobility restrictions), MSTGAT-Net showed superior adaptability. In the UK LTLA dataset, following the introduction of tiered restrictions in October 2020, MSTGAT-Net adapted to the changing patterns within 3-4 days, while baseline models required 7-10 days to adjust their predictions accordingly.

\subsection{Regional Variation in Performance}

When analyzing performance across different types of regions, several patterns emerged that provide further insights into MSTGAT-Net's capabilities.

\subsubsection{Population Density Effects}

We categorized regions in the Japan dataset into high, medium, and low population density areas and found that MSTGAT-Net's improvement over baseline models varied across these categories. The largest improvements were observed for high-density regions (15.3\% lower MAE than Graph WaveNet) and medium-density regions (11.8\% lower MAE), while improvements for low-density regions were more modest (7.5\%).

This pattern suggests that modeling spatial dependencies provides particularly significant benefits for densely connected areas with complex transmission dynamics, where traditional time series approaches are most limited. The adaptive graph learning is especially valuable in these complex urban environments where fixed graph structures may not adequately capture the multi-faceted relationships.

\subsubsection{Geographical Heterogeneity}

Analysis of region-specific performance across the UK LTLA dataset revealed that MSTGAT-Net offered the largest improvements in areas with high mobility and complex inter-regional connections. Metropolitan areas like London, Manchester, and Birmingham saw the most substantial improvements (average 18.2\% reduction in MAE compared to Graph WaveNet), further highlighting the value of adaptive spatial modeling in complex urban environments.

\subsection{Computational Efficiency Analysis}

While achieving superior forecasting accuracy, MSTGAT-Net maintains reasonable computational efficiency compared to baseline models.

\subsubsection{Training Time Comparison}

The modular architecture and parameter-efficient designs (such as the low-rank attention approximation) result in competitive training times. For the Japan dataset (47 regions), MSTGAT-Net required approximately 1.2x the training time of Graph WaveNet and 0.9x the training time of ASTGCN. For the larger Regional dataset (785 regions), the relative efficiency improved further, with MSTGAT-Net requiring only 1.1x the training time of Graph WaveNet, demonstrating good scalability.

\subsubsection{Inference Efficiency}

For inference, MSTGAT-Net exhibits comparable efficiency to baseline models. The average inference time per batch on the Japan dataset was 12.3ms for MSTGAT-Net, compared to 10.1ms for Graph WaveNet and 14.8ms for ASTGCN. This efficiency makes MSTGAT-Net practical for real-time epidemic forecasting applications, even when frequent updates are required.

\subsection{Limitations and Practical Considerations}

Despite MSTGAT-Net's strong performance, several limitations and practical considerations should be acknowledged.

\subsubsection{Very Large Graphs}

While the low-rank approximations in the adaptive graph attention improve efficiency compared to standard attention mechanisms, the computational complexity still increases significantly with the number of nodes. For extremely large graphs (tens of thousands of nodes), further optimizations would be necessary.

\subsubsection{Interpretability Challenges}

Although the attention weights provide some level of interpretability, explaining specific predictions remains challenging due to the complex interactions between multiple model components. Future work could focus on enhancing the model's explainability through techniques like integrated gradients or counterfactual explanations.

\subsubsection{Limited Multivariate Capability}

The current model primarily focuses on univariate forecasting (case counts). Extending it to multivariate forecasting with additional features (e.g., hospitalizations, deaths, testing rates) would require architectural modifications to capture dependencies between different variables.

\subsection{Conclusion}

The comprehensive evaluation of MSTGAT-Net demonstrates its effectiveness for spatiotemporal epidemic forecasting across diverse datasets and forecast horizons. The ablation studies confirm the value of each architectural innovation, with the adaptive graph attention module providing the largest contribution to performance, followed by the dilated multi-scale temporal module and the progressive prediction module.

The model's ability to learn meaningful spatial relationships without explicit geographical information, adapt to changing dynamics during regime changes, and maintain high accuracy for longer forecast horizons makes it particularly valuable for epidemic monitoring and response planning. By addressing several limitations of existing approaches, MSTGAT-Net contributes to more accurate and reliable spatiotemporal forecasting, which is crucial for informed decision-making in public health and resource allocation.

Future work will focus on extending the model to handle multivariate data, incorporate external covariates, provide uncertainty estimates, and scale efficiently to even larger spatial networks.
